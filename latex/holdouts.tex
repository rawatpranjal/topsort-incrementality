\section*{Randomized User Holdouts}

This analysis estimates the causal effect of advertising by leveraging a randomized experiment implemented by the online marketplace. The marketplace assigned a portion of its user base to a holdout or control group that was withheld from receiving advertising impressions. This section details the methodology used to identify the experimental groups, construct the analytical dataset, estimate the treatment effect, and interpret the results.

\subsection*{Detection}

The first step was to identify the members of the marketplace-run control group. The defining characteristic of these users was their exclusion from ad impressions despite being active purchasers. The identification procedure began with a universe of all 4,926,305 purchasing users. A weekly attrition algorithm then iterated through 28 weeks of impression data; any user from the initial universe found to have an impression in any week was removed from the candidate set. A final cleaning step removed 3,164 holdout-assigned users who nonetheless registered click activity, resulting in a final control group of 780,969 users. The remaining 4,142,172 purchasers form the treatment group.

The experiment was active for the entire sample period. To structure the analysis, the data was partitioned into Period 1 (2025-03-10 to 2025-06-30) and Period 2 (2025-07-01 onward). The analytical cohort was defined as the 1,119,128 users who made at least three purchases in Period 1. For this cohort, a rich set of control variables ($X$) were engineered by aggregating each user's activity in Period 1, including average weekly revenue, purchases, and clicks, as well as user tenure and vendor-related metrics such as vendor variety and spend concentration. Four distinct outcome variables ($Y$) were defined based on user behavior in Period 2: total revenue, total purchases, distinct vendors, and spend concentration.

\subsection*{Model}

To estimate the causal effects, we employ a Double Machine Learning (DML) framework for an Interactive Regression Model (IRM), which allows the treatment effect to be fully heterogeneous across covariates. The primary causal estimands are the Average Treatment Effect (ATE), $\theta_0 = E[g(1, X) - g(0, X)]$, and the Group Average Treatment Effect (GATE), $\theta_{\mathcal{G}} = E[g(1, X) - g(0, X) | X \in \mathcal{G}]$ for a subgroup $\mathcal{G}$. The DML model is estimated independently for each of the four outcome variables.

DML uses machine learning to model two nuisance functions: the conditional outcome model, $\mu(D, X) = E[Y | D, X]$, and the propensity score model, $m(X) = P(D=1 | X)$. In our implementation, these are estimated using an `LGBMRegressor` and `LGBMClassifier`, respectively. The core of DML is the use of a Neyman-orthogonal score function, which provides a debiased estimate of the treatment effect. To mitigate biases from overfitting, the procedure uses cross-fitting. This framework provides robust, $\sqrt{n}$-consistent estimates of the causal parameters, enabling valid statistical inference.

\subsection*{Results}

The analysis reveals four distinct, statistically significant, and positive impacts of advertising on user behavior. The overall Average Treatment Effects (ATEs) for each dimension are presented in Table \ref{tab:ate_summary}.

\begin{table}[htbp!]
\centering
\caption{Summary of Average Treatment Effects (ATEs)}
\label{tab:ate_summary}
\begin{tabular}{lrr}
\toprule
Outcome Variable & Absolute Lift (ATE) & Relative Lift (\\%)
\\
\midrule
Revenue (\\) & +\$36.34 & +43.17\%
\\
Purchases (\#) & +0.76 & +39.74\%
\\
Vendor Variety (\#) & +0.13 & +57.74\%
\\
Spend Concentration (%) & +0.06 & +54.69\%
\\
\bottomrule
\end{tabular}
\end{table}

To investigate the drivers of these average effects, Group Average Treatment Effects (GATEs) were estimated across quantiles of the Period 1 behavioral covariates. The results for the revenue outcome are presented in Table \ref{tab:gate_revenue}. The effect is positive and significant for nearly all user segments, but is largest in absolute terms for the highest-revenue users (Q4) and largest in relative terms for the lowest-revenue users (Q1).

\begin{table}[htbp!]
\centering
\caption{Heterogeneous Effects (GATEs) on Revenue}
\label{tab:gate_revenue}
\begin{tabular}{llrrr}
\toprule
Dimension & Subgroup & Baseline Revenue & ATE (\$ Lift) & ATE (\% Lift) \\
\midrule
\multirow{5}{*}{P1 Revenue} & Q1 (Lowest) & \$49.70 & \$28.29 & +57.0\%
\\
 & Q2 & \$67.75 & \$23.77 & +35.1\%
\\
 & Q3 & \$95.95 & \$18.72 & +19.5\%
\\
 & Q4 & \$210.44 & \$23.25 & +11.1\%
\\
 & Q5 (Highest) & \$432.60 & \$102.49 & +23.7\%
\\
\bottomrule
\end{tabular}
\end{table}

\subsection*{Interpretation}

The analysis produces a robust, multi-dimensional estimate of the causal impact of advertising. The primary conclusion is unequivocal: advertising is a powerful and profitable driver of incremental value, fundamentally changing how users shop, not just how much they spend. The four causal effects—increasing overall revenue, purchase frequency, vendor discovery, and loyalty—provide a holistic view of advertising effectiveness.

The heterogeneity analysis is the most actionable part of this study. It reveals two key strategic insights. First, there is a \textit{persuadable window}: the effect of advertising is strongest on users newest to the platform and decays for more tenured users, for whom the incremental effect on revenue eventually becomes statistically insignificant. This implies that the highest return on investment is generated by targeting users early in their lifecycle. Second, advertising has an \textit{amplifier effect}. The relative (percentage) lift is highest for users who were previously low-value, but the absolute dollar lift is highest for users who were already high-value. This suggests a dual purpose for advertising: activating the long tail of less-engaged customers while simultaneously extracting even more value from top-tier users.