\section*{Introduction}

Online marketplaces have become a dominant force in e-commerce, and with their growth, sponsored search and display advertising have emerged as a critical revenue stream and a key channel for vendors to reach potential customers. This paper investigates the causal effectiveness of such advertising within a large US-based online marketplace. Our central objective is to determine the causal impact of advertising clicks on consumer purchase behavior. This is a central question for vendors whose advertising profitability hinges on this relationship.

The setting for our study is an automated advertising ecosystem managed by an ad-tech platform. When a user searches for a product or navigates the marketplace, a real-time auction occurs in milliseconds to determine which ads are displayed. Vendors, acting as advertisers, do not bid manually in these high-frequency auctions. They instead create campaigns with specified budgets and target return on investment (ROI), and an automated bidding (``autobidding") system places bids on their behalf. The autobidding algorithm applies budget pacing to prevent budget exhaustion while remaining competitive. The ad-tech platform employs a scoring mechanism to rank ads, where the score is a product of the vendor's bid and a quality score, the latter typically being a predicted click-through rate (pCTR) for the specific user and ad. This pCTR is often obtained from machine learning algorithms.  Due to a fixed number of available ad slots only the top-ranking ads are displayed to the user. Some of these result in an impression or view. The business model in this case is pay-per-click (PPC), meaning vendors are charged only when a user clicks on their ad. And the amount they pay is determined by the auction payment rule i.e. a first or second price auction. The platform uses both types of auctions. 

This context presents significant challenges for causal inference. The primary challenge is to isolate the persuasive effect of an ad click from the user's pre-existing purchase intent. A user who is already inclined to buy a product is more likely to click on an ad for it, leading to a spurious correlation between clicks and purchases. High-intent users may be more likely to click as well as purchase. Vendors who are perceived as being of a higher quality may drive both higher clicks and purchases. Periods of promotions or discounts may drive both clicks and purchases. While we do know exactly what happens prior to a click, the user's journey after a click is not well understood. A user might purchase the clicked product, a different product from the same vendor (a “halo” purchase), or return to the platform to make a purchase at a later date. The platform's internal attribution model, which assigns a purchase to the first click within a 14-day window, is a business accounting rule and does not necessarily reflect a causal link.

This paper seeks to answer the following research question: How effective are ads for vendors? Do some vendors stand to gain more or less? What is their incremental return on ad spend? What is the causal effect of an advertising click on a consumer's propensity to purchase? We address this question by leveraging a granular dataset that provides a complete mapping of auctions, bids, impressions, and clicks. We employ two-way fixed-effects and mixed-effects models to control for unobserved confounding and heterogeneity and to identify the causal parameters of interest. Our contribution is to provide a detailed, micro-level analysis of a large-scale, real-world advertising system.
