\section*{Literature Review}
\label{sec:literature}

The literature can be broadly categorized by the identification strategy employed: experimental studies that rely on randomization, and observational studies that use econometric methods.

A prominent stream of research uses Randomized Controlled Trials (RCTs) to obtain clean identification. \citet{blake2015consumer}, in a landmark series of experiments at eBay, establish two foundational results for paid search: brand-keyword advertising is largely ineffective for a well-known brand due to nearly perfect substitution from organic search results, while non-brand advertising is effective only for new and infrequent users. This highlights the important role of consumer heterogeneity. \citet{lewis2014online} demonstrate that the primary impact of display advertising may occur offline and that the vast majority of the sales lift comes from users who view (but do not click on) the ad. Methodologically, their work is also notable for using the RCT result as a specification check to validate a difference-in-differences model, bridging the gap between experimental and non-experimental approaches. Further experimental work focusing on market structure by \citet{golden2019effects} reveals that even in a direct duopoly, the competitive spillovers from one firm suspending its search advertising are surprisingly minimal, questioning the "prisoner's dilemma" narrative often used to justify defensive brand-keyword bidding. These experimental findings consistently show that the true causal effects of advertising are often far smaller than naive correlations would suggest.

When large-scale randomization is not feasible, researchers turn to econometric models of observational data. \citet{shapiro2021tv} analyze 288 consumer packaged goods brands and find that TV advertising elasticities are extremely small, leading to the conclusion that over 80\% of brands have a negative marginal return on investment (ROI), suggesting widespread over-investment. Their identification strategy relies on the institutional features of the ad buying process combined with a rich set of fixed effects. A similar high-dimensional fixed effects approach is used by \citet{du2019advertising}, who analyze the effect of advertising on brand attitudes rather than sales, finding that different media channels have distinct effects on different attitude dimensions. These studies showcase the power of modern panel data methods in handling endogeneity. However, \citet{gordon2023comparison} provide a crucial note of caution by directly comparing observational methods against the "ground truth" from hundreds of Facebook RCTs. They find that even state-of-the-art machine learning models fail to replicate the experimental results, producing estimates that are severely biased upward, diagnosing this as a fundamental "data problem."

A related stream of work models the dynamic interplay between different advertising formats. \citet{kireyev2016do}, using a Vector Error Correction Model (VEC), show that display ad exposure can significantly increase search conversions over time, but it also increases search clicks, thereby raising costs. This highlights the attribution problem where last-click models misallocate credit.

In summary, the literature yields several key insights. First, the causal effects of advertising are consistently found to be much smaller than non-experimental or naive industry metrics suggest. Second, consumer heterogeneity is paramount; advertising is most effective on new or uninformed consumers. Third, cross-channel effects are significant and dynamic, creating complex attribution challenges. Finally, a significant methodological tension exists: while observational studies with rich panel data offer valuable insights, recent evidence underscores the difficulty of fully controlling for selection bias, reinforcing the importance of credible experimental designs. 
