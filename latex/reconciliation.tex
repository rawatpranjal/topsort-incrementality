\section*{Discussion}

The vendor-week specification yields an average click elasticity of 0.77, suggesting substantial returns to advertising at the firm level. In contrast, the user-week specification produces a near-zero average elasticity (0.028) with a negative median user-specific coefficient (-0.25). The user-vendor-week logit specification reveals strong immediate effects (coefficient 1.658) but modest carryover effects (coefficient 0.176).

These apparently contradictory results admit a coherent interpretation. The positive vendor-level elasticities alongside minimal user-level effects indicate that advertising primarily reallocates purchases across vendors rather than expanding aggregate demand. The concentration of effects in the immediate period, as shown by the distributed lag model, suggests advertising accelerates existing purchase intentions rather than creating new demand.

The heterogeneity in estimated elasticities—vendor elasticities ranging from 0.19 to 1.27 and user elasticities from -0.62 to 2.51—indicates substantial variation in advertising responsiveness. This heterogeneity implies that platform-optimal advertising allocation differs substantially from a uniform policy.

The negative median user elasticity warrants particular attention. Under a selection model where high-intent users self-select into both clicking and purchasing, the negative coefficient reflects reverse causality: users with lower baseline purchase propensity require more advertising exposure before converting. This interpretation is consistent with the strong immediate effects in the user-vendor specification, where controlling for user-vendor fixed effects partially addresses this selection problem.