\section*{User-Vendor-Week Panel}

To examine advertising effectiveness at the most granular level, we construct a user-vendor-week panel focusing on the marketplace's "power players"—the users and vendors who drive the majority of economic activity. 

We employ a two-stage selection process to identify power players based on their economic significance in a pre-treatment period (May 2025). First, we identify the top vendors who collectively account for 80\% of total platform revenue. Second, among users who transacted with these top vendors, we select those whose cumulative spending represents 80\% of revenue at these vendors. This dual threshold approach ensures we focus on the most economically meaningful interactions while maintaining computational tractability.

The resulting panel spans two months (June-July 2025) and captures the complete interaction history between selected users and vendors on a weekly basis. To address the inherent sparsity of purchase events in such granular data, we employ a balanced panel approach where all potential user-vendor-week combinations are included, with missing interactions coded as zeros.

Table \ref{tab:uvw_panel_dims} summarizes the key dimensions of the constructed panel.

\begin{table}[htbp!]
\centering
\caption{User-Vendor-Week Panel Dimensions}
\label{tab:uvw_panel_dims}
\begin{tabular}{lr}
\toprule
Metric & Value \\
\midrule
Power Users (Top 80\% spenders) & 29,153 \\
Power Vendors (Top 80\% sellers) & 4,000 \\
Analysis Period & 8 weeks \\
Total Observations & 1,466,907 \\
Balanced Panel Size & 6,706,230 \\
\bottomrule
\end{tabular}
\end{table}

\subsection*{Model}

A key challenge in measuring advertising effectiveness is the temporal mismatch between exposure and conversion. Users may be influenced by ads but delay their purchase decisions, creating a lag between click and purchase events. To capture these dynamics, we specify a distributed lag fixed-effects logit model:

\begin{equation}
\Pr(\text{Purchase}_{ivt} = 1) = \Lambda(\beta_0 \cdot \text{Click}_{ivt} + \beta_1 \cdot \text{Click}_{iv,t-1} + \alpha_i + \delta_v + \gamma_t)
\end{equation}

where $\Lambda(\cdot)$ is the logistic function, $\text{Click}_{ivt}$ is a binary indicator for whether user $i$ clicked on vendor $v$'s ads in week $t$, $\text{Click}_{iv,t-1}$ is the lagged click indicator, and $\alpha_i$, $\delta_v$, and $\gamma_t$ are user, vendor, and week fixed effects, respectively.

This specification allows us to decompose the total effect of advertising into two components. First, there is an \textit{immediate effect} ($\beta_0$), which captures the impact of clicks on same-week purchases. Second, there is a \textit{carryover effect} ($\beta_1$), representing the persistent influence of past advertising exposure.

\subsection*{Results}

Table \ref{tab:uvw_distlag_results} presents the results from the distributed lag model estimated on a 50\% random sample of users to ensure computational feasibility.

\begin{table}[htbp!]
\centering
\caption{Distributed Lag Fixed-Effects Logit Model Results}
\label{tab:uvw_distlag_results}
\begin{tabular}{lc}
\toprule
 & Purchase (Binary) \\
\midrule
Had Click (Current Week) & 1.658*** \\
 & (0.0168) \\
Had Click (Lagged 1 Week) & 0.176*** \\
 & (0.0252) \\
\midrule
User Fixed Effects & Yes \\
Vendor Fixed Effects & Yes \\
Week Fixed Effects & Yes \\
Observations & 3,462,867 \\
Pseudo R² & 0.241 \\
\bottomrule
\end{tabular}
\end{table}

Note: Standard errors clustered at the user level in parentheses. *** p < 0.001. Approximately 670,623 observations dropped due to missing lagged values, and 2,572,740 observations removed due to fixed effects with no variation in outcomes.

The coefficient of 1.658 on concurrent clicks indicates that clicking on an ad increases the odds of purchasing from that vendor in the same week by approximately 425\% ($e^{1.658} - 1 \approx 4.25$). This substantial effect suggests that clicks are a strong signal of immediate purchase intent, though we cannot fully separate selection from persuasion effects even with our rich fixed effects structure.

The positive and significant coefficient on lagged clicks (0.176) demonstrates that advertising influence extends beyond the immediate period. Users who clicked on a vendor's ad in the previous week have 19\% higher odds of purchasing in the current week, even after controlling for current-period clicks. This carryover effect is consistent with models of consideration set formation and gradual persuasion.

The immediate effect is approximately 9.4 times larger than the carryover effect, suggesting that while advertising has persistent influence, its impact is heavily front-loaded.
